\section{Metodología, Herramientas, y Ambiente de Desarrollo}
En esta sección se describirán las metodologías que se usarán para gestionar el ciclo de vida del proyecto. Además, se detallarán las herramientas y el ambiente de desarrollo en que se trabajará.
\subsection{Metodología a Usar}
Para el proyecto se utilizará la metodología ágil SCRUM\cite{1}. Esta metodología fue desarrollada por los desarrolladores de software Ken Schwaber y Jeff Sutherland. El marco de trabajo Scrum consiste en los Equipos Scrum y sus roles, eventos, artefactos y reglas asociadas\cite{2}. Cada componente dentro del marco de trabajo sirve para un propósito específico y es esencial para el éxito de Scrum. El propósito de utilizar esta metodología es que al ser ágil nos ofrece una adaptabilidad a los requisitos, además que nos ofrece adaptabilidad a los requisitos y fomenta la colaboración continua entre los miembros del equipo. Esto permite responder de manera rápida a los cambios en el entorno del proyecto y a las necesidades del cliente, lo que a su vez contribuye a la entrega de productos de alta calidad en ciclos de desarrollo cortos.
\subsection{Herramientas de Desarrollo}
%Tabla de hardware y software
\subsubsection{Hardware}
Para este proyecto se cuenta con 6 computadores personales y 1 teléfono celular de prueba como se muestra en la tabla 
\ref{table:1}.
\begin{table}[H]
    \centering
    \caption{Hardware}
	\vspace{0.2cm}
    \begin{tabular}{|c|c|c|c|c|} \hline
        \textbf{PC} & \textbf{Procesador} & \textbf{Ram} & \textbf{Sistema operativo} & \textbf{Propósito}  \\ \hline
         PC1 & Intel Core i5-10300H       & 16 GB        & Windows 11 & Desarrollo \\\hline
         PC2 & Ryzen 5 2600x       &  16GB      & Windows 10 & Desarrollo\\\hline
         PC3 & Ryzen 7 4800H       &  16GB      & Windows 11  & Pruebas\\\hline
         PC4 & Ryzen 7 4800H       &  16GB      & Windows 11 & Desarrollo y Pruebas\\\hline
		 PC5 & Intel Core i5-4690k &  16GB		& Arch Linux & Desarrollo y Pruebas\\\hline
		 PC6 & MacBook Air 2015	   &  4GB		& Arch Linux & Desarrollo\\\hline
		 Celular1 &	POCO M4 Pro 5G &  6GB		& Android	 & Pruebas\\\hline
    \end{tabular}
    
    \label{table:1}
\end{table}
\newpage
\subsubsection{Software}
%Profundizar en el uso del software visual studio, mysql, php, etc etc
Para el software se contarán con los siguientes programas especificados en la tabla 2.
\begin{table}[H]
    \centering
    \caption{Software}
	\vspace{0.2cm}
    \begin{tabular}{|l|l|} \hline
        \textbf{Programa} & \textbf{Versión} \\ \hline
         Visual Studio Code & 1.85.1\\\hline
         Kotlin & 2.0.0 \\\hline
         SQLlite &  3.45.2\\\hline
         Android Studio &  2024.2.1.9\\\hline
         GitHub &  3.4.3 \\\hline
    \end{tabular}
    \label{table:2}
\end{table}

\subsection{Ambiente de Desarrollo}
El proyecto será desarrollado en las instalaciones del Departamento de Matemática y Ciencia de la Computación de la Universidad de Santiago de Chile. También se hará uso de los espacios provistos por la Biblioteca Central de la misma institución.
Las reuniones a distancia serán realizadas a través de Discord. Se almacenará la información de los avances y reuniones en Google Drive y se hará uso de GitHub para coordinar el desarrollo de la aplicación.\\
Para el progreso exitoso de este proyecto será necesario conjugar las distintas cargas académicas y responsabilidades personales de cada miembro del equipo de trabajo. El uso de la metodología ágil será de gran utilidad con su enfoque primeramente en las personas.\\
Los recursos humanos se pueden apreciar en la Tabla \ref{table:3} con los cargos de cada integrante.
\begin{table}[H]
    \centering
    \caption{Recursos Humanos}
	\vspace{0.2cm}
    \begin{tabular}{|l|l|} \hline
        \textbf{Integrante} & \textbf{Cargo}\\ \hline
		Osvaldo Soto & Product Owner\\\hline
		Gabriel Carrillo & Scrum Master\\\hline
		Erick Aranda & Team\\\hline
		Jose Cuellar & Team\\\hline
         Elías Gangas & Team\\\hline
         Vicente Rojas & Team\\\hline
    \end{tabular}
    
    \label{table:3}
\end{table}
