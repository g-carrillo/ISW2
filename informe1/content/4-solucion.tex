\section{Descripción de la Solución Propuesta}
En esta sección se detalla la solución propuesta a la problemática anteriormente definida. Se presentan las características, propósitos, alcances, y limitaciones de esta.
\subsection{Características de la Solución}
Las caracteristicas de la solución son las siguientes:

\begin{itemize}
    \item Configución de alarmas: El sistema permite configurar distintas alarmas.
    \item Desafíos para desactivar alarmas: El sistema incluira distintos desafios para poder desactivar la alarma (aritmeticos/puzzle).
    \item Dificultades de los desafíos: El sistema contendra distintos niveles de dificultad.
    \item Configuración de alarmas por intervalo: El usuario puede configurar una serie de alarmas que suenan entre un intervalo de tiempo definido y cuyos desafíos son incrementalmente más difíciles.
\end{itemize}
\subsection{Propósitos de la Solución}
La solución propuesta consiste en el desarrollo de una aplicación de alarmas innovadora que ofrece un método de despertar más efectivo y personalizado. Esta aplicación permitirá a los usuarios configurar alarmas, integrando ejercicios mentales como retos cognitivos que deben completar para desactivar la alarma. Al incorporar esta funcionalidad, se busca fomentar la actividad cerebral y asegurar un despertar más activo y consciente. Además, la aplicación recopilará datos comportamientos de los usuarios, lo que permitirá disminuir la dificultad o aumentar según el tiempo que demore en desactivar la alarma, garantizando así una experiencia de despertar más satisfactoria, efectiva y entretenida.
 
\begin{itemize}
    \item Experiencia usuario: Ofrecer un método de despertar que se adapte a las necesidades individuales de los usuarios, proporcionando una herramienta que garantice un despertar más activo y entretenido.
    \item Activación cerebral: Incorporar ejercicios cognitivos que desafíen la mente del usuario, promoviendo una mayor agilidad mental desde el momento del despertar.
    \item Obtención de datos: Recopilar información sobre el comportamiento de los usuarios con el fin de aumentar o disminuir la dificultad de los retos, mejorando la personalización de la experiencia.
\end{itemize}

\subsection{Alcances y Limitaciones de la Solución}
\begin{itemize}
    \item \textbf{Alcances:}
    \begin{itemize}
        \item Realización de una aplicación móvil para crear alarmas personalizadas con desafios de distintas dificultades.
        \item  Entrega de documentación técnica, documentación de código y manual de usuario.
    \end{itemize}
    \item \textbf{Limitaciones:}
     \begin{itemize}
        \item La aplicación solo se podrá utilizar en un dispositivo móvil Android.
        \item La aplicación solo estara disponible en español.
    \end{itemize}
    
\end{itemize}
