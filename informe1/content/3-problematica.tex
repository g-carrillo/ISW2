\section{Descripción de la Problemática}
En esta sección se da a conocer la motivación y el enfoque del proyecto, además de las soluciones actualmente presentes a la problemática.

\subsection{Motivación}
Muchas personas presentan dificultades para despertar con alarmas tradicionales. Esto puede deberse a diversos factores, como trastornos del sueño, factores ambientales, o simplemente variaciones normales entre individuos. Frecuentemente, las alarmas tradicionales son desactivadas por dichas personas mientras se encuentran en un estado de ``inercia del sueño", una etapa entre el sueño y la vigilia en que el individuo experimenta capacidad cognitiva limitada.
\\
Una solución actualmente aplicada a esta problemática es el uso de desafíos cognitivos que deber ser resueltos antes de desactivar las alarmas. Si bien esto puede dar buenos resultados, no se trata de una solución completa. Por ejemplo, no considera que la severidad de la inercia del sueño varía entre un día y otro incluso para un mismo individuo, y por lo tanto también debería variar la dificultad de los desafíos presentados. Si bien existe la posibilidad de establecer múltiples alarmas manualmente, esto presenta de la dificultad adicional de organizar manualmente dichas alarmas y sus dificultades, lo cual se dificulta aún más por los horarios de sueño irregulares presentados por personas con trastornos del sueño.
\\
Este proyecto busca desarrollar una solución completa a esta problemática, permitiendo que es usuario pueda establecer fácilmente conjuntos de alarmas cuyos desafíos se ajusten a sus necesidades personales.

\subsection{Definición del Problema}
El problema esta en que normalmente tenemos aplicaciones predeterminadas, las cuales ocupamos para poner nuestras alarmas para poder despertar, ese es el uso común de estas, pero en mas de un caso se da, que inconscientemente desactivamos esta alarma por el motivo que sea, pero seguido de esto no tenemos nada mas que nos pueda despertar, y la gente tiende a poner mas de una alarma consecutiva para que en el caso que se desactive esta misma, vuelva a sonar otra dentro de un periodo de tiempo.
\subsection{Estado del Arte}
	Las siguientes aplicaciones de alarmas ocupan nichos similares en la tienda de aplicaciones de Android.

	\begin{enumerate}
		\item[] \textbf{Alarmy:} Permite establecer alarmas con desafíos aritméticos, entre otros.
		\item[] \textbf{Walk Me Up:} Requiere que el usuario camine una cierta cantidad de pasos para desactivar la alarma.
		\item[] \textbf{AMdroid:} Permite establecer distintos desafíos, entre otras configuraciones.
	\end{enumerate}
