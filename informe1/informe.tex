\documentclass[letterpaper, 10pt]{article}
\usepackage[margin=2cm]{geometry}
\usepackage{graphicx}
\usepackage{float}

% Table de contenidos clickeable
\usepackage{hyperref}
\hypersetup{
	colorlinks,
	citecolor=black,
	filecolor=black,
	linkcolor=black,
	urlcolor=black
}


%%%% IMPORTANTE: 
%%%%	Reemplazar "PROYECTO" con el nombre del proyecto.
%%%%
\newcommand{\nomproyecto}{PROYECTO}

% Localización
\usepackage[spanish]{babel}

% Números de páginas
\pagenumbering{arabic}
\usepackage{fancyhdr}
\pagestyle{fancy}
\fancyhf{}
\renewcommand{\headrulewidth}{0pt}
\fancyhead[R]{\thepage}

%

% Ambiente para las listas de la portada
\newenvironment{simplelist}
	{
		\renewcommand\labelitemi{}
		\begin{itemize}
		\setlength{\itemsep}{0pt}
		\setlength{\parskip}{0pt}
	}
	{\end{itemize}}

%

\begin{document}

\begin{titlepage}

\begin{center}
	{\bf Universidad de Santiago de Chile}
	\\
    {\bf Facultad de Ciencia}
	\\
    {\bf Departamento de Matem\'atica y Ciencia de la Computaci\'on}
\end{center}

%

\vfill
\begin{center}
{\large\bf Ingeniería de Software II}
\\
{\Large\bf Informe Nº1. “Presentación y Planificación del Proyecto Ágil”}\\
\vspace{1cm}
{\Huge\bf \nomproyecto}
\end{center}
\vfill

%

\begin{flushright}
	\begin{minipage}{0.3\textwidth}
		Integrantes:
		\begin{simplelist}
			\item Gabriel Carrillo
			\item Erick Aranda
			\item Jose Cuellar
			\item Elías Gangas
			\item Vicente Rojas
		\end{simplelist}
	

		Profesor:
		\begin{simplelist}
			\item Dino Araya
		\end{simplelist}
	
		Fecha de entrega:
		\begin{simplelist}
			\item 15 de octubre del 2024
		\end{simplelist}
	\end{minipage}
\end{flushright}

\end{titlepage}

%

\tableofcontents
\listoftables
\listoffigures
\newpage

%

\section{Introducción}

En el presente informe se describirá en detalle el proyecto a desarrollar junto con su planificación de actividades.
\newline
Este proyecto intentar ofrecer una solución más apta para aquellas personas que experimentan un sueño inusualmente pesado, facilitando el proceso de configuración de alarmas afines a las necesidades del usuario.

\section{Objetivos del Proyecto}
En esta sección se presenta el objetivo general y los objetivos específicos para el correcto desarrollo e implementación de \nomproyecto.
\subsection{Objetivo General}
Diseñar e implementar la aplicación móvil \nomproyecto, una aplicación de alarmas que requieren la solución de desafíos para ser desactivadas.
\subsection{Objetivos Específicos}
A continuación se detallan los objetivos específicos que se consideran hitos claves para el desarrollo del proyecto con metodología ágil.
\begin{itemize}
	\item Anteproyecto y planificación.
	\item Análisis de requerimientos (Historias de usuario).
	\item Análisis orientado a objetos (AOO).
	\item Diseño orientado a objetos (DOO).
	\item Programación orientada a objetos (POO).
	\item Pruebas de programación.
	\item Pruebas de aceptación.
	\item Puesta en producción.
\end{itemize}

\section{Descripción de la Problemática}
En esta sección se da a conocer la motivación y el enfoque del proyecto, además de las soluciones actualmente presentes a la problemática.
\subsection{Motivación}
Muchas personas presentan dificultades para despertar con alarmas tradicionales. Esto puede deberse a diversos factores, como trastornos del sueño, factores ambientales, o simplemente variaciones normales entre individuos. Frecuentemente, las alarmas tradicionales son desactivadas por dichas personas mientras se encuentran en un estado de "inercia del sueño", una etapa entre el sueño y la vigilia en que el individuo experimenta capacidad cognitiva limitada.
\\
Una solución actualmente aplicada a esta problemática es el uso de desafíos cognitivos que deber ser resueltos antes de desactivar las alarmas. Si bien esto puede dar buenos resultados, no se trata de una solución completa. Por ejemplo, no considera que la severidad de la inercia del sueño varía entre un día y otro incluso para un mismo individuo, y por lo tanto también debería variar la dificultad de los desafíos presentados. Si bien existe la posibilidad de establecer múltiples alarmas manualmente, esto presenta de la dificultad adicional de organizar manualmente dichas alarmas y sus dificultades, lo cual se dificulta aún más por los horarios de sueño irregulares presentados por personas con trastornos del sueño.
\\
Este proyecto busca desarrollar una solución completa a esta problemática, permitiendo que es usuario pueda establecer fácilmente conjuntos de alarmas cuyos desafíos se ajusten a sus necesidades personales.

\subsection{Definición del Problema}
%TODO
\subsection{Estado del Arte}
	Las siguientes aplicaciones de alarmas ocupan nichos parecidos en la tienda de aplicaciones de Android.

	\begin{enumerate}
		\item[] \textbf{Alarmy:} Permite establecer alarmas con desafíos aritméticos, entre otros.
		\item[] \textbf{Walk Me Up:} Requiere que el usuario camine una cierta cantidad de pasos para desactivar la alarma.
		\item[] \textbf{AMdroid:} Permite establecer distintos desafíos, entre otras configuraciones.
	\end{enumerate}

\section{Descripción de la Solución Propuesta}
En esta sección se detalla la solución propuesta a la problemática anteriormente definida. Se presentan las características, propósitos, alcances, y limitaciones de esta.
\subsection{Características de la Solución}
\subsection{Propósitos de la Solución}
\subsection{Alcances y Limitaciones de la Solución}

\section{Metodología, Herramientas, y Ambiente de Desarrollo}
En esta sección se describirán las metodologías que se usarán para gestionar el ciclo de vida del proyecto. Además, se detallarán las herramientas y el ambiente de desarrollo en que se trabajará.
\subsection{Metodología a Usar}
Para el proyecto se utilizará la metodología ágil SCRUM\cite{1}. Esta metodología fue desarrollada por los desarrolladores de software Ken Schwaber y Jeff Sutherland. El marco de trabajo Scrum consiste en los Equipos Scrum y sus roles, eventos, artefactos y reglas asociadas\cite{2}. Cada componente dentro del marco de trabajo sirve para un propósito específico y es esencial para el éxito de Scrum. El propósito de utilizar esta metodología es que al ser ágil nos ofrece una adaptabilidad a los requisitos, además que nos ofrece adaptabilidad a los requisitos, además de fomentar la colaboración continua entre los miembros del equipo. Esto permite responder de manera rápida a los cambios en el entorno del proyecto y a las necesidades del cliente, lo que a su vez contribuye a la entrega de productos de alta calidad en ciclos de desarrollo cortos.
\subsection{Herramientas de Desarrollo}
%Tabla de hardware y software
\subsubsection{Hardware}
Para este proyecto se cuenta con 6 computadores personales y 1 teléfono celular de prueba como se muestra en la tabla 
\ref{table:1}.
\begin{table}[H]
    \centering
    \caption{Hardware}
	\vspace{0.2cm}
    \begin{tabular}{|c|c|c|c|c|} \hline
        \textbf{PC} & \textbf{Procesador} & \textbf{Ram} & \textbf{Sistema operativo} & \textbf{Propósito}  \\ \hline
         PC1 & Intel Core i5-10300H       & 16 GB        & Windows 11 & Desarrollo \\\hline
         PC2 & Ryzen 5 2600x       &  16GB      & Windows 10 & Desarrollo\\\hline
         PC3 & Ryzen 7 4800H       &  16GB      & Windows 11  & Pruebas\\\hline
         PC4 & Ryzen 7 4800H       &  16GB      & Windows 11 & Desarrollo y Pruebas\\\hline
		 PC5 & Intel Core i5-4690k &  16GB		& Arch Linux & Desarrollo y Pruebas\\\hline
		 PC6 & MacBook Air 2015	   &  4GB		& Arch Linux & Desarrollo\\\hline
		 Celular1 &	POCO M4 Pro 5G &  6GB		& Android	 & Pruebas\\\hline
    \end{tabular}
    
    \label{table:1}
\end{table}
\newpage
\subsubsection{Software}
%Profundizar en el uso del software visual studio, mysql, php, etc etc
Para el software se contarán con los siguientes programas especificados en la tabla 2.
\begin{table}[H]
    \centering
    \caption{Software}
	\vspace{0.2cm}
    \begin{tabular}{|l|l|} \hline
        \textbf{Programa} & \textbf{Versión} \\ \hline
         Visual Studio Code & 1.85.1\\\hline
         Kotlin & 2.0.0 \\\hline
         SQLlite &  3.45.2\\\hline
         Android Studio &  2024.2.1.9\\\hline
         GitHub &  3.4.3 \\\hline
    \end{tabular}
    \label{table:2}
\end{table}

\subsection{Ambiente de Desarrollo}
El proyecto será desarrollado en las instalaciones del Departamento de Matemática y Ciencia de la Computación de la Universidad de Santiago de Chile. También se hará uso de los espacios provistos por la Biblioteca Central de la misma institución.
Las reuniones a distancia serán realizadas a través de Discord. Se almacenará la información de los avances y reuniones en Google Drive.\\
Para el progreso exitoso de este proyecto será necesario conjugar las distintas cargas académicas y responsabilidades personales de cada miembro del equipo de trabajo. El uso de la metodología ágil será de gran utilidad con su enfoque primeramente en las personas.\\
Los recursos humanos se pueden apreciar en la Tabla \ref{table:3} con los cargos de cada integrante.
\begin{table}[H]
    \centering
    \caption{Recursos Humanos}
	\vspace{0.2cm}
    \begin{tabular}{|l|l|} \hline
        \textbf{Integrante} & \textbf{Cargo}\\ \hline
		Osvaldo Soto & Product Owner\\\hline
		Gabriel Carrillo & Scrum Master\\\hline
		Erick Aranda & Team\\\hline
		Jose Cuellar & Team\\\hline
         Elías Gangas & Team\\\hline
         Vicente Rojas & Team\\\hline
    \end{tabular}
    
    \label{table:3}
\end{table}

\section{Plan de Trabajo}
En esta sección se detalla la planificación de entrevistas del proyecto y se presenta la carta Gantt.
\subsection{Planificación de Entrevistas}

\subsection{Planificación del Proyecto}
Se establece una programación simple del proyecto dividido en 5 etapas principales:
\begin{itemize}
	\item \textbf{Etapa 1:} Anteproyecto. Definición y planificación del proyecto ágil.
	\item \textbf{Etapa 2:} Análisis de requerimientos y casos de uso.
	\item \textbf{Etapa 3:} Análisis Orientado a Objetos (AOO).
	\item \textbf{Etapa 4:} Diseño Orientado a Objetos (DOO).
	\item \textbf{Etapa 5:} Programación Orientada a Objetos (POO).
\end{itemize}
%
A continuación se presenta la planificación del proyecto mediante la carta Gantt correspondiente.
%
\begin{figure}[h]
	\centering
	\includegraphics[width=\textwidth]{img/CartaGantt.png}
	\caption{Carta Gantt}
	\label{fig:CartaGantt}
\end{figure}
\clearpage

\section{Conclusiones}



\clearpage
\begin{thebibliography}{a}
    \bibitem{1} \textbf{Schwaber, K.} (1997). \textit{Scrum development process}, in \textit{Business Object Design and Implementation: Scrum development process Proceedings 16 October 1995, Austin, Texas}, pp. 117-134. Springer.
     \bibitem{2}    \textbf{Scrum Alliance}. (2020). \textit{La Guía de ScrumTM}. 
\end{thebibliography}

\end{document}
 
