\section{Análisis Orientado a Objetos}
Se describen a continuación las tres etapas generales en que se divide el análisis orientado a objetos bajo la metodología OMT++. Cada etapa se contruye a partir de una etapa anterior o de las especificaciones de casos de uso obtenidos durante la creación del avance anterior.

El análisis de objetos consiste en la especificación de los conceptos clave relacionados al sistema siendo desarrollado. A partir de un análisis de las descripciones de casos de uso se produce un "modelo de objetos" siguiendo las normas del lenguaje unificado de modelado o UML.

En la etapa de análisis de compotamiento se definen las operaciones realizadas por el usuario para la manipulación e ingreso de información, sin especificar detalles de la interfaz. Este artefacto recibe el nombre de "lista de especificación de operaciones" y se crea con el objetivo de que el sistema final puedo ejecutar todas las operaciones contenidas en dicha lista. Similar a la etapa anterior, gran parte de del trabajo realizado en esta etapa consiste en el análisis de responsabilidad del usuario dentro de las descripciones de los casos de uso en el avance anterior.

\subsection{Análisis de Objetos}

\subsubsection{Modelo de Objetos de Análisis}

\subsection{Análisis de Comportamiento}
\subsubsection{Diccionario de Datos del Modelo de Objetos}
\begin{table}[H]
    \centering
    \caption{Lista de Operaciones de Usuario}
    \vspace{10pt}
    \begin{tabular}{}
        \hline
	
        \hline
    \end{tabular}
    
    \label{table:1}
\end{table}


\subsection{Análisis de Comportamiento}
\subsubsection{Especificación de Operaciones}
\begin{table}[H]
    \centering
    \caption{Lista de Operaciones de Usuario}
    \vspace{10pt}
    \begin{tabular}{| p{0.4\linewidth} | p{0.4\linewidth} |}
        \hline
	
        \hline
    \end{tabular}    
    
    \label{table:1}
\end{table}

