\section{Análisis Orientado a Objetos}
Se describen a continuación las tres etapas generales en que se divide el análisis orientado a objetos bajo la metodología OMT++. Cada etapa se contruye a partir de una etapa anterior o de las especificaciones de casos de uso obtenidos durante la creación del avance anterior.

El análisis de objetos consiste en la especificación de los conceptos clave relacionados al sistema siendo desarrollado. A partir de un análisis de las descripciones de casos de uso se produce un "modelo de objetos" siguiendo las normas del lenguaje unificado de modelado o UML.

En la etapa de análisis de compotamiento se definen las operaciones realizadas por el usuario para la manipulación e ingreso de información, sin especificar detalles de la interfaz. Este artefacto recibe el nombre de "lista de especificación de operaciones" y se crea con el objetivo de que el sistema final puedo ejecutar todas las operaciones contenidas en dicha lista. Similar a la etapa anterior, gran parte de del trabajo realizado en esta etapa consiste en el análisis de responsabilidad del usuario dentro de las descripciones de los casos de uso en el avance anterior.

\subsection{Análisis de Objetos}
Durante el análisis de objetos se utilizan los resúmenes de los casos de uso detalladas en el avance anterior. A partir de estos se han indentificado las posibles clases para el modelo de objetos de análisis. Se considera cada caso de uso para luego evaluar modelos apropiados al sistema a desarrollar.

\subsubsection{Modelo de Objetos de Análisis}
\begin{figure}[H]
	\centering
	\includegraphics[page=1,width=\textwidth]{./img/uml.png}
	\caption{Diagrama de Objetos de Análisis}
        \vspace{10pt}
	\label{fig:Diagrama de Objetos de Análisis}
\end{figure}

\subsubsection{Diccionario de Datos del Modelo de Objetos}
\begin{table}[H]
    \centering
    \caption{Diccionario de Datos}
    \vspace{10pt}
    \begin{tabular}{|l|p{0.4\linewidth}|p{0.4\linewidth}|}
        \hline
        \textbf{Clase} & \textbf{Descripción} & \textbf{Atributos} \\
        \hline
        Desafío & Desafío a resolver & Dificultad: nivel de dificultad\\
        \hline
        Desafío Aritmético & Desafío que requiere la resolución de problemas aritméticos & Números: número de operandos\newline Operaciones Aritméticas: operadores a utilizar\newline Solución: solución del desafío\newline Dificultad: nivel de dificultad del desafío\\
        \hline
        Desafío Rompecabezas & Desafío que requiere la resolución de un rompecabezas. & Piezas: número de piezas a organizar \newline Solución: orden correcto de las piezas\newline Dificultad: nivel de dificultad del desafío\\
        \hline
        Desafío Memorice & Desafío que requiere un juego de memorice. & Cartas: cartas a memorizar\newline Solución: solución del desafío\newline Dificultad: nivel de dificultad del desafío\\
        \hline
        Alarma & Una alarma o grupo de alarmas fijadas. & Sonido: sonido a utilizar\\
        \hline
        Alarma Única & Alarma que no pertenece a un grupo. & Hora: hora de activación de la alarma\newline Sonido: sonido a utilizar\\
        \hline
        Alarma Por Intervalo & Grupo de alarmas que sonarán durante un intervalo de tiempo. & Hora Inicio: hora de la primera alarma\newline Hora Final: hora de la última alarma\newline Cantidad: número de alarmas que sonarán durante el intervalo.\newline Sonido: sonido a utilizar\\
        \hline

    \end{tabular}
    
    \label{table:1}
\end{table}


\subsection{Análisis de Comportamiento}
Se utilizan los casos de uso definidos anteriormente para crear la lista de operaciones. A partir de la descripción de flujo normal para dichos casos se definen las operaciones realizadas por el usuario, identificando todos los casos de uso donde se repiten.

\subsubsection{Especificación de Operaciones}
Se listan a continuación todas las operaciones identificadas a partir de los casos de uso.
\begin{table}[H]
    \centering
    \caption{Lista de Operaciones de Usuario}
    \vspace{5pt}
    \begin{tabular}{| c | l | l |}
        \hline
        Número & Operación & ID Caso de Uso \\
        \hline
        1 & Ingresar hora de inicio & 1 - 12 \\
        \hline
        2 & Ingresar hora de término & 1 - 12 \\
        \hline
        3 & Ingresar frecuencia & 1 - 12 \\
        \hline
        4 & Seleccionar opción ``Desafío aritmético" & 1 - 2 - 10 - 12 - 6 \\
        \hline
        5 & Seleccionar opción ``Desafío rompecabezas" & 1 - 2 - 10 - 12 - 6 \\
        \hline
        6 & Seleccionar opción ``Desafío memorice" & 1 - 2 - 10 - 12 - 6 \\
        \hline
        7 & Seleccionar opción ``Ingresar Intervalo" & 1 \\
        \hline
        8 & Ingresar hora de alarma & 2 - 10 - 6 \\
        \hline
        9 & Seleccionar opción ``Ingresar alarma" & 2 \\
        \hline
        10 & Ingresar solución & 3 \\
        \hline
        11 & Seleccionar opción ``Ingresar Solución" & 3 \\
        \hline
        12 & Mover piezas del rompecabezas & 4 \\
        \hline
        13 & Seleccionar opción ``Comprobar" & 4 \\
        \hline
        14 & Seleccionar cartas & 5 \\
        \hline
        15 & Seleccionar opción ``Desactivar alarma" & 7 \\
        \hline
        16 & Seleccionar opción ``Alarma única" & 8 \\
        \hline
        17 & Seleccionar opción ``Alarma por Intervalos" & 8 \\
        \hline
        18 & Seleccionar opción "Seleccionar alarma" & 9 \\
        \hline
        19 & Seleccionar opción "Eliminar alarma" & 8 \\
        \hline
        20 & Ingresar invervalo & 10 - 12 \\
        \hline
        21 & Seleccionar opción "Guardar cambios" & 10 - 6 - 12 \\
        \hline
    \end{tabular}    
    
    \label{table:2}
\end{table}
\newpage
\subsection{Especificación de la interfaz de usuario}
En esta sección se presenta el diagrama de diálogo y la especificación de sus componentes que servirán como base para la interfaz de usuario de Snoozefest. El diagrama de diálogo representa la forma en que el usuario interactúa con el sistema y se mueve a través de los diálogos, mientras que la especificación de sus componentes establece los elementos que compondrán dichos diálogos.

\subsubsection{Diagramas de Diálogo}
El diagrama de diálogo permite visualizar claramente la interacción entre el usuario y el sistema, permitiendo visualizar las operaciones que conectan a los diálogos.
\begin{figure}[H]
	\centering
	\includegraphics[page=1,width=\textwidth]{./img/dialogos.pdf}
	\caption{Diagramas de Diálogo}
        \vspace{10pt}
	\label{fig:Diagrama de Diálogos}
\end{figure}
\newpage
\subsubsection{Especificación de Componentes}
Habiendo establecido los diálogos, se especifican los elementos que los componen.

\begin{figure}[H]
	\centering
	\includegraphics[width=0.3\textwidth]{~/uni/inge2/informe3/img/componentes/01-PrincipalSnoozefest.png}
	\caption{Snoozefest}
        \vspace{5pt}
	\label{fig:Snoozefest}
\end{figure}

\begin{figure}[H]
	\centering
	\includegraphics[width=0.5\textwidth]{~/uni/inge2/informe3/img/componentes/02-CrearAlarmaUnica.png}
	\caption{Crear Alarma Única}
        \vspace{5pt}
	\label{fig:Crear Alarma Única}
\end{figure}

\begin{figure}[H]
	\centering
	\includegraphics[width=0.5\textwidth]{~/uni/inge2/informe3/img/componentes/03-CrearAlarmaPorIntervalos.png}
	\caption{Crear Alarma por Intervalos}
        \vspace{5pt}
	\label{fig:Crear Alarma por Intervalos}
\end{figure}

\begin{figure}[H]
	\centering
	\includegraphics[width=0.5\textwidth]{~/uni/inge2/informe3/img/componentes/04-ModificarAlarmaUnica.png}
	\caption{Modificar Alarma Única}
        \vspace{5pt}
	\label{fig:Modificar Alarma Única}
\end{figure}

\begin{figure}[H]
	\centering
	\includegraphics[width=0.5\textwidth]{~/uni/inge2/informe3/img/componentes/05-ModificarAlarmaIntervalos.png}
	\caption{Modificar Alarma por Intervalos}
        \vspace{5pt}
	\label{fig:Modificar Alarma por Intervalos}
\end{figure}

\begin{figure}[H]
	\centering
	\includegraphics[width=0.5\textwidth]{~/uni/inge2/informe3/img/componentes/06-BorrarAlarma.png}
	\caption{Borrar Alarma}
        \vspace{5pt}
	\label{fig:Borrar Alarma}
\end{figure}

\begin{figure}[H]
	\centering
	\includegraphics[width=0.3\textwidth]{~/uni/inge2/informe3/img/componentes/07-DesactivarAlarma.png}
	\caption{Desactivar Alarma}
        \vspace{5pt}
	\label{fig:Desactivar Alarma}
\end{figure}

\begin{figure}[H]
	\centering
	\includegraphics[width=0.5\textwidth]{~/uni/inge2/informe3/img/componentes/08-DesafioRompecabezas.png}
	\caption{Desafío Rompecabezas}
        \vspace{5pt}
	\label{fig:Desafío Rompecabezas}
\end{figure}

\begin{figure}[H]
	\centering
	\includegraphics[width=0.5\textwidth]{~/uni/inge2/informe3/img/componentes/09-DesafioMemorice.png}
	\caption{Desafío Memorice}
        \vspace{5pt}
	\label{fig:Desafío Memorice}
\end{figure}

\begin{figure}[H]
	\centering
	\includegraphics[width=0.5\textwidth]{~/uni/inge2/informe3/img/componentes/10-DesafioAritmetico.png}
	\caption{Desafío Aritmético}
        \vspace{5pt}
	\label{fig:Desafío Aritmético}
\end{figure}

\begin{figure}[H]
	\centering
	\includegraphics[width=0.5\textwidth]{~/uni/inge2/informe3/img/componentes/11-ModificarAlarma.png}
	\caption{Modificar Alarma}
        \vspace{5pt}
	\label{fig:Modificar Alarma}
\end{figure}

\begin{table}[H]
    \centering
    \caption{Tabla de descripción de componentes}
    \vspace{5pt}
\begin{tabular}{|p{4cm}|p{5cm}|p{5cm}|}
    \hline
    \textbf{Diálogo} & \textbf{Manipulación} & \textbf{Retroalimentación} \\ \hline
    Ingresar alarma por intervalos &
    Ingresar hora de inicio\newline
    Ingresar hora de término\newline
    Ingresar cantidad de alarmas\newline
    Seleccionar tipo de desafío\newline
    Seleccionar opción "Ingresar Alarma"
    &
    Entrada:\newline
    Hora de Inicio\newline
    Hora de Término\newline
    Cantidad de Alarmas\newline
    Tipo de Desafío\newline
\\ \hline
    Ingresar alarma única &
    Ingresar hora\newline
    Seleccionar tipo de desafío\newline
    Seleccionar opción "Ingresar Alarma"
    &
    Entrada:\newline
    Hora\newline
    Tipo de Desafío\newline
 \\ \hline
    Modificar alarma por intervalo &
    Ingresar hora de inicio\newline
    Ingresar hora de término\newline
    Ingresar cantidad de alarmas\newline
    Seleccionar tipo de desafío\newline
    Seleccionar opción "Modificar Alarma"
    &
    Entrada:\newline
    Hora de Inicio\newline
    Hora de Término\newline
    Cantidad de Alarmas\newline
    Tipo de Desafío\newline
    
 \\ \hline
    Modificar alarma única &
    Ingresar hora\newline
    Seleccionar tipo de desafío\newline
    Seleccionar opción "Modificar Alarma"
    &
    Entrada:\newline
    Hora\newline
    Tipo de Desafío\newline
 \\ \hline
    Resolver aritmética &
    Ingresar solución ejercicio\newline
    &
    Entrada:\newline
    Solución\newline
 \\ \hline
    Resolver rompecabezas &
    Ingresar movimiento de piezas
    &
    Entrada:\newline
    Movimiento de piezas\newline
    Solución\newline

 \\ \hline
    Resolver memorice &
    Ingresar cartas seleccionadas
    &
    Entrada:\newline
    2 Cartas seleccionadas\newline
    Salida:

 \\ \hline
    Borrar alarma &
    Seleccionar Alarma\newline
    Seleccionar opción "Eliminar Alarma"
    &
    Entrada:\newline
    ID de la alarma\newline
 \\ \hline
    Modificar alarma &
    Seleccionar tipo de alarma\newline
    &
    Entrada:\newline
    Tipo de Alarma\newline
 \\ \hline
    Snoozefest &
    Seleccionar opción "Crear Alarma"\newline
    Seleccionar opción "Tipo de alarma"\newline
    &
    \\ \hline
\end{tabular}
    
    \label{table:2}
\end{table}
