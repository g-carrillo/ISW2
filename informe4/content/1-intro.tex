\section{Introducción}
El presente informe tiene como finalidad ofrecer una visión clara y preliminar del trabajo realizado en la etapa de diseño orientado a objetos, conforme a la metodología OMT++ aplicada al proyecto Snoozefest en curso. En este contexto, se presenta el cuarto informe de avance, en el cual se exponen las principales actividades desarrolladas, los artefactos generados y las consideraciones metodológicas adoptadas en esta fase del proceso de desarrollo de software.
El documento se organiza en torno a la aplicación de un diseño que integra el paradigma Modelo-Vista-Controlador(MVC), orientado a la obtención de un modelo de clases que facilite la estructuración coherente de las funcionalidades y el comportamiento del sistema, de manera que cada capa cumpla su rol específico dentro de la arquitectura propuesta. Asimismo, se describen las trazas de eventos que contribuyen a la identificación de los métodos y las interacciones entre las clases, logrando así una caracterización precisa del comportamiento esperado.
La información contenida en las secciones posteriores busca incrementar la comprensión del diseño orientado a objetos en OMT++, presentando los elementos conceptuales y técnicos de forma sistemática y coherente. De esta manera, el lector podrá obtener una perspectiva integral de las decisiones tomadas, las justificaciones técnicas y el fundamento metodológico subyacente, factores que resultan esenciales para comprender la base conceptual y operativa del diseño a implementar.