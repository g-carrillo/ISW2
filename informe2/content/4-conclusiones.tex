\section{Conclusiones}
La recolección de requisitos con el cliente es uno de los pasos más cruciales en el desarrollo de software, ya que permite comprender con claridad las expectativas y necesidades del usuario final. En este caso específico de la aplicación de alarmas, las funcionalidades clave como la creación, modificación, eliminación y desactivación de alarmas deben ser detalladamente definidas para evitar malentendidos y asegurar que el software cumpla con los requerimientos establecidos.

El análisis de los casos de uso desempeña un papel esencial en este proceso. Al especificar detalladamente cómo deben comportarse las distintas funcionalidades de la aplicación desde la perspectiva del usuario, se logra una alineación entre las expectativas del cliente y las capacidades del sistema. Con una especificación detallada de cada caso de uso, se puede avanzar de manera efectiva hacia el análisis orientado a objetos (AOO), donde se establecerán las relaciones entre los componentes del software. Este enfoque sistemático sienta las bases para un diseño sólido y un desarrollo exitoso del sistema.