\section{Introducción}

En el presente informe el equipo organiza las principales ideas obtenidas durante el avance anterior. Se capturan los requerimientos funcionales, no funcionales, y de implementación. A partir de estos se crea el diagrama de caso de uso UML y se especifican los casos de uso no triviales, es decir, aquellos cuya complejidad no ameritan una especificación detallada. En la especificación, se describe el flujo de normal de dichos casos de uso, además de las excepciones que puedan ocurrir durante ellos.
